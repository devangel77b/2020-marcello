\documentclass[11pt]{wrceletter}


\name{Dennis Evangelista}
\position{Assistant Professor}
%\email{\href{mailto:evangeli@usna.edu}{\emph{evangeli@usna.edu}}}
\email{evangeli@usna.edu}
\telephone{410-293-6132}

\date{\today}

\usepackage{designature}
\signature{\vspace*{-0.5in}\includesignature\\%
Dennis Evangelista '99\\%
Course 2 \& Course 6 (SB and MEng)\\%
MIT-WHOI Joint Program 2004-2007\\%
PhD, Integrative Biology, UC Berkeley\\%
PE}
%\signature{Dennis Evangelista\\Assistant Professor} % title not needed if in letterhead
\address{\null} %{105 Maryland Avenue\\Annapolis, MD 21402} % leave blank, provided in letterhead
%\longindentation=0in % to change signature to be flushleft

\begin{document}
\begin{letter}{% recipient address here
MIT Department of Electrical Engineering and Computer Science\\
77 Massachusetts Avenue 38-401\\
Cambridge, MA 02139}
%360 Woods Hole Road MS31, Clark 223\\
%Woods Hole, MA 02543}

% opening here
\opening{To the Admissions Committee:}
\raggedright % if you like this sort of thing
\setlength{\parindent}{15pt} % if you like this sort of thing

I am happy to recommend MIDN 1/C Ethan Marcello. I have known Mr.~Marcello for 3 years; he was an outstanding student in my ES200 (Introduction to Systems Engineering) section, covering embedded microcontrollers, C and Matlab, and our first offering of ES281C (School of Drones, quadrotors). Additionally, I have enjoyed working with him as his honors capstone advisor and academic advisor. Mr.~Marcello is an outstanding student and would do very well in your program. 

Mr.~Marcello has excelled as an honors Robotics and Control Engineering major here at the Naval Academy. Our departmental focus includes control, autonomy, system dynamics, and automatic control.  I have tried to push Mr.~Marcello at the levels I would expect of MIT course 6 undergraduates and he has done an excellent job in all respects. 

Mr.~Marcello's honors project is to perform extreme maneuvers using quadrotors; a specific goal is to replicate display dives in Anna's Hummingbirds (\emph{Calypte anna}), in which males dive and perform 9G pullout maneuvers in attempt to win matings with females. Beyond that particular species of hummingbird, Mr.~Marcello is also broadly interested in how to make systems perform other extreme maneuvers, which could include other dances and displays, movement through obstacles and apertures, perching, escape and evasion. Marcello has been pushing himself to study maneuvers in animals, to do experimentation with actual quadrotor hardware under computer control and tracking, and to learn theory, beyond what is normally covered in undergraduate controls, regarding motion planning, underactuated systems, learning, and systems with extreme agility. 

Mr.~Marcello aims to be an engineering duty officer (EDO) in the Navy. EDOs design and build ships and submarines, lead research projects, and apply new technologies to naval ship and weapon systems. Having worked on new submarine projects myself as a Navy lieutenant and Naval Reactors engineer, I believe he will do as excellent a job in this role as he has in every other task I've seen him perform. Mr.~Marcello is an outstanding candidate. His ability to juggle the tasks thrown at him as an honors student and a young naval officer show he will navigate MIT well. 

\closing{Very respectfully,} % provides empty 

%\ps{post script here}
%\encl{enclosure here}
\end{letter}
\end{document}


%The following are my goals based on what I've heard/read I needed to do, and what I think best suits my interests:
%M.S. degree program (especially at MIT) that allows me to spend most of my time in robotics, autonomy, and controls. (Also interested in embedded systems and AI/deep learning)
%Obtain a Navy p-code upon completion of aforementioned program.
%Satisfy graduate school/NPS requirement for Navy EDOs with aforementioned program.
%Here are some of the websites that have directed me toward thinking an Electrical Engineering program:
%
%USNA OSP information cite
%USNA literature on subspecialty codes (i.e. they're required for OSP admittance)
%Navy Personnel Command on EDO reqs
%Navy Personnel Command on P-codes (scroll to p-codes section, and then the ESRs for each code are tabled towards the bottom).
%Subspecialty Code 560XP (Mechanical Engineering) <--- Included as a quick reference. This one has a lot of very specific courses requiring maritime applications (e.g. ESR-2, ESR-4, ESR-5, ESR-8, ESR-10)
%Subspecialty Code 53XXP (EE) <-- This one looks easier to manage, which is why I said I'm looking more at a EECS sort of program.
%
%If you require more information just let me know, but this is mostly what I've dug up about what I need to do in order for USNA to let me go to grad school (and then also make sure my degree can validate my EDO grad school requirement, although that isn't as important because in the worst case scenario they'll just send me to NPS to take classes that will make them satisfied.)
%
%Attached below is also my resume. I've added some bullets on the second page that aren't officially part of my resume, but may help you in writing the letter of recommendation. Thank you for doing this for me!
